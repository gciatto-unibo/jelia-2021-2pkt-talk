% !TeX spellcheck = en_GB
%
\documentclass[presentation]{beamer}
\mode<presentation>{\usetheme{AMSCesenaPurpleAndGold}}
\setbeamertemplate{bibliography item}{\insertbiblabel}
%%%%%%%%%%%%%%%%%%%%%%%%%%%%%%%%%%%%%%%%%%%%%%%%%%%%%%%%%%%%%%%%%%%%%%%%%%%%%%%%
%
\usepackage{jelia-2021-2pkt-talk}
%%%%%%%%%%%%%%%%%%%%%%%%%%%%%%%%%%%%%%%%%%%%%%%%%%%%%%%%%%%%%%%%%%%%%%%%%%%%%%%%
\title[Lazy Stream Manipulation in Prolog]{
    % same title of the presented paper
    Lazy Stream Manipulation in Prolog via Backtracking: The Case of \twopkt{}
}
%
% \subtitle{Extended Abstract}
%
% same authors order of the presented paper
\author[Ciatto et al.]{
	\emph{Giovanni Ciatto}$^{*}$ % empth the presenting author
	\and 
	Roberta Calegari$^{\dagger}$
	\and
	Andrea Omicini$^{*}$
}
%
\institute[UniBo]{
    $^{*}$Dipartimento di Informatica -- Scienza e Ingegneria (DISI)
    \\
    $^{\dagger}$Alma Mater Research Institute for Human-Centered Artificial Intelligence (Alma AI)
    \\
    \textsc{Alma Mater Studiorum} -- Università di Bologna
    \\
    \texttt{
        \{\emph{giovanni.ciatto}, roberta.calegari, andrea.omicini\}@unibo.it % emph the presenting author's email
    }
}
%
\date[JELIA, 2021]{
	$17^{th}$ Edition of the European Conference on
	\\
	Logics in Artificial Intelligence (JELIA 2021)
	\\
	May 17, 2021, Klagenfurt, Austria (Online Event)
}
%%%%%%%%%%%%%%%%%%%%%%%%%%%%%%%%%%%%%%%%%%%%%%%%%%%%%%%%%%%%%%%%%%%%%%%%%%%%%%%%
\AtBeginSection[]
{
%\\\\\\\\\\\\\\\\\\\\\
\begin{frame}<beamer>[c,noframenumbering]
\frametitle{Next in Line\ldots}
\tableofcontents[sectionstyle=show/shaded,subsectionstyle=hide]
\end{frame}
%\\\\\\\\\\\\\\\\\\\\\
}
\AtBeginSubsection[]
{
%\\\\\\\\\\\\\\\\\\\\\
\begin{frame}<beamer>[shrink,noframenumbering]
    \frametitle{Focus on\ldots}
	\mbox{~}
	\tableofcontents[currentsubsection,sectionstyle=shaded,subsectionstyle=show/shaded]
	\mbox{~}
\end{frame}
%\\\\\\\\\\\\\\\\\\\\\
}
%%%%%%%%%%%%%%%%%%%%%%%%%%%%%%%%%%%%%%%%%%%%%%%%%%%%%%%%%%%%%%%%%%%%%%%%%%%%%%%%
\begin{document}
%%%%%%%%%%%%%%%%%%%%%%%%%%%%%%%%%%%%%%%%%%%%%%%%%%%%%%%%%%%%%%%%%%%%%%%%%%%%%%%%

%\\\\\\\\\\\\\\\\\\\\\
\frame{\titlepage}
%\\\\\\\\\\\\\\\\\\\\\

%===============================================================================
\section{Motivation \& Context}
%===============================================================================

%\\\\\\\\\\\\\\\\\\\\\
\begin{frame}[allowframebreaks]{Context}
    
    \begin{itemize}
        \item We live in the data-driven AI era
        %
        \begin{itemize}
            \item pervasive need of processing wide amounts of data
            \item[eg] coming form sensors, smart devices, social networks etc.
        \end{itemize}
        
        \bigskip
        
        \item Stream processing is the preferred way towards \alert{scalability}
        %
        \begin{itemize}
            \item streams as possibly \alert{unlimited FIFO queues} of data
            \item either \emph{cold} (a.k.a. \alert{pull}) or \emph{hot} (a.k.a. \alert{push}) depending data \alert{generation}
            \item processing does not require loading \emph{all} data in memory
            \item support for \alert{lazy} manipulation of data
        \end{itemize}
        
        \framebreak
        
        \item Main-stream programming languages are blending FP with OOP
        %
        \begin{itemize}
            \item[$\rightarrow$] cascades of higher-order functions as stream-processing pipelines
            \item[eg] map, flat map, filter, fold, reduce, etc 
        \end{itemize}
        
        \bigskip
        
        \item Plenty of technologies for stream processing \alert{in the large}
        %
        \begin{itemize}
            \item featuring advancend, time-related operators 
            \item[eg] Kafka, Flink, Spark, Storm, etc.
        \end{itemize}
        
    \end{itemize}
\end{frame}
%\\\\\\\\\\\\\\\\\\\\\

%\\\\\\\\\\\\\\\\\\\\\
\begin{frame}[c]{Motivation}
    \begin{block}{Question 1}
        \centering
        What is the role of LP in stream processing?
    \end{block}
    %
    \begin{itemize}
        \item LP as a means to express \alert{complex event processing} \ccite{AnicicFRSSS10,AnicicRFS12}
        \item \alert{ASP} as a means to reason over \emph{hot} streams \ccite{BeckEB17,Beck2018}
    \end{itemize}

    \vfill

    \begin{block}{Question 2}
        \centering
        What is the role of \emph{Prolog} in stream processing?
    \end{block}
    %
    \begin{itemize}
        \item[$\uparrow$] this is the focus of this work
    \end{itemize}
\end{frame}
%\\\\\\\\\\\\\\\\\\\\\

%\\\\\\\\\\\\\\\\\\\\\
\begin{frame}{Contributions of this Work}

    \begin{enumerate}
        \item We show that Prolog \emph{already} supports lazy stream manipulation
        %
        \begin{itemize}
            \item as Prolog solvers can be intended as \alert{prosumers} of data
        \end{itemize}
        
        \vfill
        
        \item We propose a notion of \alert{generator} aimed letting solvers:
        %
        \begin{itemize}
            \item lazily consume streams of data from the external world
        \end{itemize}

        \vfill  
        
        \item We sketch \alert{state-machine-based} design for Prolog solvers
        %
        \begin{itemize}
            \item aimed at supporting the aforementioned notion of \alert{generator}
            \item without affecting its syntax, nor its semantics
        \end{itemize}

        \vfill  

        \item We demonstrate how generators can be exploited in practice
        %
        \begin{itemize}
            \item[eg] letting Prolog invoke external solvers and lazily consume their responses
            \item via the \twopkt{} technology \ccite{homepage2PKt}
        \end{itemize}
    \end{enumerate}

\end{frame}
%\\\\\\\\\\\\\\\\\\\\\

%===============================================================================
\section{Theory / modelling / design}
%===============================================================================

%\\\\\\\\\\\\\\\\\\\\\
\begin{frame}%[allowframebreaks]
\frametitle{Theory / modelling / design}

    Provide 2-3 slides discussing the Theory / modelling / design

\end{frame}
%\\\\\\\\\\\\\\\\\\\\\

\section{Case study / Experiments / Results}

%\\\\\\\\\\\\\\\\\\\\\
\begin{frame}%[allowframebreaks]
\frametitle{Case study / Experiments / Results}

    Provide 2-3 slides discussing the Case study / Experiments / Results of the paper

\end{frame}
%\\\\\\\\\\\\\\\\\\\\\

\section{Conclusions \& future works}

%\\\\\\\\\\\\\\\\\\\\\
\begin{frame}%[allowframebreaks]
\frametitle{Conclusions \& future works}

\begin{block}{Summing up}
    Summarise the most relevant contributions of this study:
    %
    \begin{itemize}
        \item conclusion 1
        \item conclusion 2
        \item conclusion 3
    \end{itemize}
\end{block}

\begin{exampleblock}{Future works}
    Sketch some future research directions
    %
    \begin{itemize}
        \item future work 1
        \item future work 2
    \end{itemize}
\end{exampleblock}

(may be split into 2 slides)

\end{frame}
%\\\\\\\\\\\\\\\\\\\\\

%===============================================================================
\section*{}
%===============================================================================
\frame{\titlepage}

%===============================================================================
\section*{\bibname}
%===============================================================================

\setbeamertemplate{page number in head/foot}{}
%\\\\\\\\\\\\\\\\\\\\\
\begin{frame}[t,allowframebreaks,noframenumbering]\frametitle{\refname}
% \begin{frame}[c]\frametitle{\refname}
	\footnotesize
%	\scriptsize
    \bibliographystyle{plain}
	\bibliography{jelia-2021-2pkt-talk}
\end{frame}
%\\\\\\\\\\\\\\\\\\\\\

%%%%%%%%%%%%%%%%%%%%%%%%%%%%%%%%%%%%%%%%%%%%%%%%%%%%%%%%%%%%%%%%%%%%%%%%%%%%%%%%
\end{document}
%%%%%%%%%%%%%%%%%%%%%%%%%%%%%%%%%%%%%%%%%%%%%%%%%%%%%%%%%%%%%%%%%%%%%%%%%%%%%%%%
